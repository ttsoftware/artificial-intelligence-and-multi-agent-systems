\documentclass[12pt]{article}
\usepackage[a4paper, hmargin={2.5cm, 2.5cm}, vmargin={2.5cm, 2.5cm}]{geometry}

\usepackage[nottoc,numbib]{tocbibind}

\usepackage[utf8]{inputenc}
\usepackage[english]{babel}
\usepackage{amssymb}
\usepackage{amsfonts}
\usepackage{amsmath}
\usepackage{setspace}
\usepackage{algorithm}
\usepackage[noend]{algpseudocode}

\usepackage{tikz}
\usetikzlibrary{positioning,shapes, shadows, arrows, automata}

\usepackage{xcolor}
\usepackage{listings}
\usepackage{graphicx}
\usepackage[hidelinks]{hyperref}
\usepackage{float}
\usepackage[english]{varioref}
\usepackage{multirow}
\usepackage{hhline}
\usepackage{etoolbox}
\usepackage{seqsplit}

\usepackage{fancyhdr}

\setlength\parindent{0pt}
\usepackage[parfill]{parskip}

\definecolor{mygray}{rgb}{0.9451,0.9451,0.9451}
\lstset{
  backgroundcolor=\color{mygray},
  basicstyle=\footnotesize\ttfamily,
  mathescape,
  breaklines=true,
  numbers=left,
  numberstyle=\ttfamily,
  stepnumber=1,
  firstnumber=1,
  numberfirstline=true,
  postbreak=\raisebox{0ex}[0ex][0ex]{\ensuremath{\color{red}\hookrightarrow\space}},
  literate={->}{$\rightarrow$}{2}
           {ε}{$\varepsilon$}{1}
}

\linespread{1.3}

\title{
  \vspace{4cm}
  \begin{flushleft}
  \Large{\textbf{Warmup Assignment}} \\
  \large{Artificial Intelligence and Multi Agent Systems}
  \end{flushleft}
  \vspace{0cm}
  \begin{flushleft}
  \small
  \textit{\today}
  \end{flushleft}
  \vspace{12cm}
  \begin{flushleft}
  \small
  Troels Thomsen \texttt{152165} \\
  Rasmus Haarslev \texttt{152175} \\
  \end{flushleft}
}

\date{
	%
}

\begin{document}

\clearpage
\pagenumbering{gobble}
\thispagestyle{empty}
\maketitle

\newpage

\section{Exercise 1 (Seach Strategies)}

The results for these exercises can be seen in figure~\ref{benchmark-results}.


\subsection{a}
\label{sub:a}

The shortest length is 19, which we know because BFS will always find the shortest path if implemented correctly. BFS looks at all possible solutions and chooses the best solution.


\subsection{b}
\label{sub:b}

Running SAD2.lvl using the BFS strategy, we run out of memory before it finds any solutions.
This level is much more complex, since adding extra boxes increases the number of states exponentially, as each box and agent can be on each field in the level.


\subsection{c}
\label{sub:c}

The implementation can be seen in the \texttt{StrategyDFS.java} file, and the benchmarks can be seen in figure~\ref{benchmark-results}


\subsection{d}
\label{sub:d}

We designed a custom level with the following layout.

\begin{verbatim}
+++++++
+aAAAa+
+A 0 A+
+a A a+
+A A A+
+++++++
\end{verbatim}

This particular layout contains a lot of boxes to be moved around, with several goal states to be achieved simultaneously. This creates a very large solution space for BFS to explore in order to find the best solution. DFS on the other hand, can find a solution to this problem much more quickly, since almost any path will lead to a solution.

\begin{figure}[H]
    \begin{tabular}{|c|c|c|c|c|c|}
        \hline
        Level & Client & Time & Memory Used & Solution Length & Nodes Explored \\
        \hline
        SAD1 & BFS & 0.13 sec & 9.30MB & 19 & 78 \\
        \hline
        SAD1 & DFS & 0.10 sec & 6.20MB & 27 & 44 \\
        \hline
        SAD2 & BFS & - & - & - & - \\
        \hline
        SAD2 & DFS & 7.81 sec & 570.94MB & 5781 & 8799 \\
        \hline
        friendOfDFS & BFS & - & - & - & - \\
        \hline
        friendOfDFS & DFS & 0.20 sec & 6.82MB & 24 & 24 \\
        \hline
        friendOfBFS & BFS & 0.20 sec & 5.58MB & 2 & 17 \\
        \hline
        friendOfBFS & DFS & - & - & - & - \\
        \hline
    \end{tabular}
    \caption{Benchmark}
    \label{benchmark-results}
\end{figure}

\end{document}
